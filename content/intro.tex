\csname documentclass\endcsname[../main.tex]{subfiles}
\usepackage[utf8x]{inputenc}
\usepackage{blindtext}
\usepackage{float}
\usepackage{graphicx}
\usepackage{siunitx}
\usepackage{cite}
\usepackage{hyperref}
\usepackage{cleveref}

\begin{document}

CLICK is a mission with the overarching goal of demonstrating a crosslink between two CubeSats, both of which with a < 2U payload. 
It aims to demonstrate a miniaturized optical transmitter with capabilities of \(\ge \qty[per-mode = symbol,per-symbol = p]{10}{\mega\bit\per\second}\) optical downlinks to a \(\qty{30}{\centi\meter}\) optical ground telescope. 
The reason CLICK-A is identified as a risk-mitigating mission is because is because it is the first step in being able to prove a \(\ge \qty[per-mode = symbol,per-symbol = p]{20}{\mega\bit\per\second}\) inter-satellite optical crosslink system \cite{cierny_testing_2020}.
This mission addresses the rising need for higher and higher downlink rates as more and more data-intensive components are sent flying, such as a cameras used for hyperspectral imaging.


The CLICK missions build on top of, both, the Laser Communications Relay Demonstration (LCRD) and the Lunar Laser Communications Demonstrations (LLCD) which both showcased the feasibility and efficiency of laser communications. 
As mentioned earlier, the increase in high data collecting instruments has created a need for more efficient and powerful communications means, which has naturally pushed the traditional radio frequency (RF) based communications to a more modern optical based one.
Since RF is used as the primary mode of communication with the telemetry, tracking, and command (TT\&C), as well as, the main data downlink communication, transmitting at higher rates would require more power, more bandwidth, and sometimes higher frequencies \cite{click_a}.
This paper aims to do a deep dive into the CLICK mission itself to further understand the mission objectives and the various aspects that went into making this mission possible.


\end{document}